\documentclass[../../master.tex]{subfiles}

\begin{document}
\section{The Algebra of Complex Numbers}
\subsection{Arithmetic Operations}

% Problem 1.1.1
\begin{problem}
    Find the values of
    \[
        (1 + 2i)^{3}, \quad \frac{5}{-3 + 4i}, \quad \left(\frac{2 + i}{3 - 2i}\right)^{2}, \quad (1 + i)^{n} + (1 - i)^{n}.
    \]
\end{problem}

\begin{solution}
    These are all relatively simple calculations.
    \begin{align*}
        (1 + 2i)^3 &= (1 + 2i)(1 + 2i)(1 + 2i) \\
                  &= (-3 + 4i)(1 + 2i) \\
                  &= -11 - 2i
    \end{align*}
    \begin{align*}
        \frac{5}{-3 + 4i} &= \frac{5(-3 - 4i)}{(-3 + 4i)(-3 - 4i)} \\
                          &= \frac{-15 - 20i}{25} \\
                          &= -\frac{3}{5} - \frac{4}{5}i
    \end{align*}
    \begin{align*}
        \left(\frac{2+i}{3 - 2i}\right)^2 &= \left(\frac{(2+i)(3+2i)}{(3-2i)(3+2i)}\right)^2 \\
                              &= \left(\frac{4 + 7i}{13}\right)^2 \\
                              &= \frac{-33 + 56i}{169}
    \end{align*}
    For the final calculation, note that the number $a_n = (1 + i)^{n} + (1 - i)^{n}$ is twice the real part of $(1 + i)^{n}$ due to using the conjugate.
    Furthermore, by Euler's identity we find $(1 + i)^{n} = \sqrt{2}^{n} e^{in\pi/4}$ which has real part $\cos(n\pi/4)$. Thus
    \[
        (1 + i)^{n} + (1 - i)^{n} = 2\sqrt{2}^{n} \cos\frac{n\pi}{4}
    \]
    Note that this can be shown via modulo arguments but it is far more tedious.
\end{solution}

% Problem 1.1.2
\begin{problem}
    If $z = x + iy$ ($x$ and $y$ real), find the real and imaginary parts of
    \[
        z^4, \quad \frac{1}{z}, \quad \frac{z -1}{z + 1}, \quad \frac{1}{z^2}
    \]
\end{problem}

\begin{solution}
    Again, these are just computations.
    \begin{align*}
        (x+iy)^4 &= \left((x+iy)(x+iy)\right)^2 \\
                &= \left((x^2 - y^2) + 2xyi\right)\left((x^2-y^2) + 2xyi\right) \\
                &= (x^4 - 6x^2y^2 + y^4) + (4x^3y - 4xy^3)i
    \end{align*}
    \begin{align*}
        \frac{1}{x+iy} &= \frac{x - iy}{x^2 + y^2} \\
                       &= \frac{x}{x^2 + y^2} - \frac{y}{x^2 + y^2}i
    \end{align*}
    \begin{align*}
        \frac{(x-1) + iy}{(x+1) + iy} &= \frac{\left((x-1) + iy\right)\left((x+1) - iy\right)}{\left((x+1) + iy\right)(\left((x+1) - iy\right)} \\
                                      &= \frac{(x^2 + y^2 - 1) + 2iy}{x^2 + 2x + y^2 +1}
    \end{align*}
    \begin{align*}
        \frac{1}{(x+iy)^2} &= \frac{1}{(x^2 - y^2) + 2xyi} \\
                           &= \frac{(x^2 - y^2) - 2xyi}{((x^2 - y^2) + 2xyi)((x^2 - y^2) - 2xyi)} \\
                           &= \frac{(x^2 - y^2) - 2xyi}{x^4 - 2x^2y^2 + y^4}
    \end{align*}
\end{solution}
% Problem 1.1.3
\begin{problem}
    Show that
    \[
        \left(\frac{-1 \pm i \sqrt{3}}{2}\right)^3 = 1 \quad \text{and} \quad \left(\frac{\pm 1 \pm i \sqrt{3}}{2}\right)^{6} = 1
    \]
    for all combinations of signs.
\end{problem}

\begin{solution}
    I'll only do one combination for each but it's easy to verify the others.
    \begin{align*}
        \left(\frac{-1 + i \sqrt{3}}{2}\right)^{3} &= \left(\frac{-2 - 2i\sqrt{3}}{4}\right) \left(\frac{-1 + i \sqrt{3}}{2}\right) \\
                                 &= \frac{2 + 6}{8} \\
                                 &= 1
    \end{align*}
    \begin{align*}
        \left(\frac{-1 + i\sqrt{3}}{2}\right)^{6} &= \left(\left(\frac{-1 +i\sqrt{3}}{2}\right)^{3}\right)^{2} \\
                                &= 1^2 \\
                                &= 1
    \end{align*}
\end{solution}

\subsection{Square Roots}

% Problem 1.2.1
\begin{problem}
    Compute
    \[
        \sqrt{i}, \qquad \sqrt{-i}, \qquad \sqrt{1 + i}, \qquad \sqrt{\frac{1 - i\sqrt{3}}{2}}.
    \]
\end{problem}

\begin{solution}
    We find $x^2 = \frac{1}{2} (0 + \sqrt{1}) = \frac{1}{2}$ so $x = \pm \frac{\sqrt{2}}{2}$. 
    Similarly, $y^2 = \pm \frac{\sqrt{2}}{2}$.
    With the relation $2xy = 1$, we have a final solution of
    \[
        \sqrt{i} = \pm \frac{\sqrt{2}}{2} \pm \frac{\sqrt{2}}{2} i
    \]
    
    We have a similar set of solutions for $\sqrt{-i}$ but now the product $2xy = -1$ so the signs of corresponding solutions are switched. Thus, we have
    \[
        \sqrt{-i} = \pm \frac{\sqrt{2}}{2} \mp \frac{\sqrt{2}}{2} i
    \]
\end{solution}
\end{document}
